\documentclass{article}
\usepackage{amsmath}
\title {Segundo documento \LaTeX}
\author{Guillermo Arag\'on}
\date{\today}

\begin{Document}
  \maketitle

  Este es mi primer docuemtno de \LaTeX

  \section{Primera Ecuaci\'on}
  Entorno de ecuaci\'on dentro de la l\'inea de texto: $\sum^{\infty}_{n=1}$

  Con el s\'imbolo \^\ se escriben lso super\'indices y con \_\ los sub\'indices.
  
  \section{Entornos de ecuaciones}

  Hay varios tipos de entornos de ecuaci\'on. Los m\'as importantes son:

  \begin{enumerate}
  \item \emph{equation}
  \begin{enumerate}
    \item \emph{equation} simple
    \item \emph{split} equation  
    
  \end{enumerate}
  \item \emph{align}  
  \end{enumerate}

  \begin{equation}
  \frac{d\hat{\theta}}{dt} = \sum_{k_{t}=1}^nC^{*}_{n}(k_{t}) 
  \end{equation}
  
  \begin{equation*}
  \frac{d\hat{\theta}}{dt} = \sum_{k_{t}=1}^nC^{*}_{n}(k_{t}) 
  \end{equation*}

  \begin{align}
    P_{c}=& \pi D \nonumber \\
         =& 2\pi r 
  \end{align}

  \begin{equation}
  \begin{split}
    P_{c}=& \pi D \\
         =& 2\pi r 
  \end{split}
  \end{equation}

\begin{subequations}
  \begin{align}
    P_{c}=& \pi D \\
         =& 2\pi r 
  \end{align}
\end{subequations}


\end{document}

