\documentclass[preprint,12pt]{revtex4}
\usepackage{grffile}%so the file kind is the last .*
\usepackage{bm} %for bold math
\usepackage{amsmath,mathrsfs}
\newcommand{\op}[1]{{\hat{\mathcal#1}}}
%\AtEndDocument{\message{^^JLaTeX Info: Executing hook `AtEndDocument'.}}
\def\colore{red}
\usepackage[usenames,dvipsnames]{xcolor}
\usepackage{animate}
%%%% fancy header
\usepackage{fancyhdr}
\pagestyle{fancy}
\renewcommand{\headrulewidth}{2pt}
%%%%%
\usepackage[spanish,english]{babel}
\usepackage[utf8]{inputenc}
%\usepackage{showkeys}
\allowdisplaybreaks[1]%biutiful equation breaker!!!
\usepackage{graphicx}
\usepackage{fancyvrb}
%%%%*****************%%%% fine hyperef 
\usepackage[backref,pdffitwindow,colorlinks,citecolor={red},linkcolor={blue}]{hyperref}
%%%%*****************%%%%
% Definitions 
\input{definitions}
%%%%%%%%%%%%%%%%%%%%%%%%
\renewcommand{\theenumi}{\arabic{enumi}}
\renewcommand{\theenumii}{.\arabic{enumii}}
\renewcommand{\theenumiii}{.\arabic{enumiii}}
%
\renewcommand{\labelenumi}{\arabic{enumi}}
\renewcommand{\labelenumii}{\arabic{enumi}.\arabic{enumii}}
\renewcommand{\labelenumiii}{\arabic{enumi}.\arabic{enumii}.\arabic{enumiii}}
%
\newcommand{\cita}{\addtocounter{enumii}{1}}
%\selectlanguage{spanish}
\usepackage{setspace}
\usepackage{lastpage}
\cfoot{Page~\thepage~of \pageref{LastPage}}
\lhead{}
\rhead{semiejes}
%\usepackage{pageslts}
\begin{document}

%Se me ocurri� otra forma de proceder, m�s algebr�ica:
\begin{enumerate}
\item	 Sea el eigenvector complejo es $\vec v=\vec v'+\vec v''$
\item	 Eso significa que el campo es \\
	  $ \vec E(t)\propto \vec v'\cos\omega t+\vec v''\sin\omega t$
\item	 Tomando componentes\\
	   $E_x=v'_x\cos\omega t+v''_x\sin\omega t$\\
	   $E_y=v'_y\cos\omega t+v''_y\sin\omega t$
\item	 Eliminando $\cos\omega t$\\
	   $v'_y E_x-v'_x E_y=(v'_y v''_x-v'_x v''_y)\sin\omega t$
\item	 y eliminando $\sin\omega t$\\
	   $v''_y E_x-v''_x E_y=(v''_y v'_x-v''_x v'_y)\cos\omega t$
\item	 De donde podemos despejar\\ 
	   $\sin\omega t=\frac{v'_y E_x-v'_x E_y}{v'_y v''_x - v'_x v''_y}$\\
	   $\cos\omega t=-\frac{v''_y E_x-v''_x E_y}{v'_y v''_x - v'_x v''_y}$
\item	 Con $\cos^2+\sin^2=1$, podemos eliminar al tiempo escribiendo\\
	   $\vec E^T M \vec E=1$
\item	 donde escribimos $\vec E$ como vector columna y $M$ es una
	   matriz con componentes\\
	   $M_{xx}=|v_y|^2/D$\\
	   $M_{xy}=-(v'_x v'_y+v''_x v''_y)/D$\\
	   $M_{yy}=|v_x|^2/D$\\
	   con\\
	   $D=(v'_y v''_x-v'_x v''_y)^2.$
\item	 La ecuacion secular para esta matriz es\\
	   $\lambda^2 -\lambda \mathrm{tr}(M) + \mathrm{det}(M)=0$\\
	   con soluciones\\
	     $\lambda_\pm=\big(\mathrm{tr}(M)\pm\sqrt{\mathrm{tr}(M)^2-4\mathrm{det}(M)}\big)/2$\\
	   a las que corresponden eigenvectores $|+\rangle$ y $|-\rangle$
	   (no confundir con polarizaciones circulares derecha e izquierda).
\item	 Escribiendo $\vec E=E_+|+\rangle + E_-|-\rangle$ obtenemos\\
	    $\vec E^T M \vec E=\lambda_- E_-^2 + \lambda_+ E_+^2 = 1$\\
	    que es la ecuacion de una elipse con semieje mayor\\
	    $a=1/\sqrt\lambda_-$\\
	    y semieje menor\\
	    $b=1/\sqrt\lambda_+$
\item	 Las direcciones correspondientes a estos semiejes son\\
	    $\tan\psi_-=(\lambda_- - M_{xx})/M_{xy}$\\
	    $\tan\psi_+=(\lambda_+ - M_{xx})/M_{xy}$
\item	 Para verificarlo hice un programita en gnuplot (anexo).
\end{enumerate}
\section{Diagonalization}
Eigenvalues and eigenvectors of a $2\times 2$ symmetric matrix
$M\ket{\lambda_i}=\lambda_i\ket{\lambda_i}$, with 
\begin{equation}\label{h.1}
\left(\begin{array}{cc} 
M_{xx}&M_{xy}\\ 
M_{xy}&M_{yy}
\end{array}\right)
.
\end{equation} 
Then,
\begin{eqnarray}\label{h.1}
\left|\begin{array}{cc}
M_{xx}-\lambda&M_{xy}\\
M_{xy}&M_{yy}-\lambda
\end{array}\right|
&=&(M_{xx}-\lambda)(M_{yy}-\lambda) -M^2_{xy}
\nonumber\\
&=&
\lambda^2-\lambda(M_{xx}+M_{yy})+M_{xx}M_{yy}-M^2_{xy}
\nonumber\\
&=&
\lambda^2-\lambda 
\mathrm{Tr}[M]
+
\mathrm{Det}[M]
=0
\nonumber\\
\to\lambda_{\pm}&=&
\frac{1}{2}\left(
\mathrm{Tr}[M]
\pm\sqrt{(\mathrm{Tr}[M])^2-4 \mathrm{Det}[M]}
\right)
.
\end{eqnarray}
The eigenvectors follow from
\begin{eqnarray}\label{h.1}
\left(\begin{array}{cc}
M_{xx}-\lambda&M_{xy}\\
M_{xy}&M_{yy}-\lambda
\end{array}\right)
\left(\begin{array}{c}
a\\
b
\end{array}\right)
&=&0
\nonumber\\
(M_{xx}-\lambda)a+M_{xy}b&=&0
\nonumber\\
M_{xy}a+(M_{yy}-\lambda)b&=&0
\nonumber\\
\to b&=&\frac{\lambda-M_{xx}}{M_{xy}}a,
\nonumber\\
\to
 \bfV
&=&
\frac{a}{M_{xy}}\left(\begin{array}{c}
M_{xy}\\
\lambda-M_{xx}
\end{array}\right)
,
\end{eqnarray}
and use $a$ to normalize the vector, then
\begin{equation}\label{h.3}
 \bfv_\pm
=
\frac{1}{\sqrt{|M_{xy}|^2+|\lambda_\pm-M_{xx}|^2}}
\left(\begin{array}{c}
M_{xy}\\
\lambda_\pm-M_{xx}
\end{array}\right)
,
\end{equation}
with $\bfv_\pm\cdot\bfv_\pm^T=1$,
and $\bfv_\mp\cdot\bfv_\pm=0$ can be easily verified for the case of
real $M$. If $M$ is complex the complex eigenvectors are not
necessarily perpendicular.  
%%%%%%%%%%%%
\end{document}

From mochan@fis.unam.mx Wed Mar  5 13:44:15 2014
Date: Tue, 4 Mar 2014 13:46:36 -0600
From: Luis Mochan <mochan@fis.unam.mx>
To: Bernardo Mendoza Santoyo <bms@cio.mx>
Subject: Re: ayuda

Si quieres probamos hoy a las 14:00 de aqu�.
Acabo de intentar entender algo de los �ngulos from scratch. Te cuento:
Considera un eigenvector complejo \vec v=\vec v'+\vec v''. Conforme
transcurre el tiempo, su punta recorre la figura \vec(t)=Re \vec v
e^{-iwt} =  \vec v' \cos wt + \vec v'' \sin wt. Considera un vector
real fijo unitario \hat n=(cos \theta, \sin \theta). Considera el
producto escalar dependiente del tiempo \alpha(t)=\hat n\cdot\vec v(t). Lo que
hice fue encontrar para qu� tiempo t se extremiza el producto escalar,
lo cual depende de la direcci�n de \hat n. A continuaci�n, busqu� para
qu� direcci�n \theta se extremiza el extremos del producto escalar. Mi
resultado es

\tan(2\theta)=2 \frac{v_x'v_y'+v_x'' v_y''}
		     {|v_x|^2 - |v_y|^2}

Lo prob� en algunos casos sencillos. Si v_y=0 obtenemos 2\theta=0,
\pi. Lo mismo si v_x=0. Si v_x=1 y v_y=i (polarizaci�n circular)
tenemos \tan2\theta=0/0, indefinido. Si v_x=v_y tenemos \theta=\pm
pi/4. Etc.

As� que parece que esta f�rmula s� da los ejes principales de la
elipse de polarizaci�n. No he hecho la talacha para ver si corresponde
con tu �ngulo Me sospecho que no coincide con tu f�rmula en general,
pero s� coincide cuando el eigenvector es real. Me sospecho que esto
debe estar desarrollado en los libros de elipsometr�a.

Saludos,
Luis
